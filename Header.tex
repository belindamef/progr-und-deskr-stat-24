% type setting
% ------------------------------------------------------------------------------
\usepackage[german]{babel}     

% fonts
% ------------------------------------------------------------------------------
\usefonttheme{professionalfonts}

% slide title and horizontal line
% ------------------------------------------------------------------------------
\setbeamertemplate{frametitle}{%
    \vskip-30pt \color{black}\large%
    \begin{minipage}[b][23pt]{120mm}%
    \flushleft\insertframetitle%
    \end{minipage}%
}

\setbeamertemplate{headline}										
{
\vskip10pt\hfill\hspace{3.5mm} 										 
\vskip15pt\color{black}\rule{\textwidth}{0.4pt} 					 
}

% slide number
% ---------------------------------------------------------------
\setbeamertemplate{navigation symbols}{}
\setbeamertemplate{footline}
{
\vskip5pt
\vskip2pt
\makebox[123mm]{\hspace{7.5mm}
\hfill Programmierung und Deskriptive Statistik $\vert$ 
Belinda Fleischmann $\vert$ 
Folie \insertframenumber}
\vskip4pt
}

% block color scheme
% ------------------------------------------------------------------------------
% colors
\definecolor{white}{RGB}{255,255,255}
\definecolor{grey}{RGB}{235,235,235}
\definecolor{middlegrey}{RGB}{210,210,210}
\definecolor{darkgrey}{RGB}{105,105,105}
\definecolor{lightgrey}{RGB}{245,245,245}
\definecolor{linkblue}{HTML}{4F86AB}
\definecolor{darkblue}{RGB}{51,51,153}
\definecolor{darkcyan}{RGB}{5,92,92}
\definecolor{middlecyan}{RGB}{0,153,153}
\definecolor{darkgreen}{RGB}{0,102,51}
\definecolor{plum}{RGB}{128,0,128}


% definitions and theorems
\setbeamercolor{block title}{fg = black, bg = grey}
\setbeamercolor{block body}{fg = black, bg = lightgrey}

% general line spacing 
% ------------------------------------------------------------------------------
\linespread{1.3}

% local line spacing
% ------------------------------------------------------------------------------
\usepackage{setspace}

% colors
% -----------------------------------------------------------------------------
\usepackage{color}
\usepackage{xcolor}

% justified text
% ------------------------------------------------------------------------------
\usepackage{ragged2e}
\usepackage{etoolbox}
\apptocmd{\frame}{}{\justifying}{}

% bullet point lists
% -----------------------------------------------------------------------------
\setbeamertemplate{itemize item}[circle]
\setbeamertemplate{itemize subitem}[circle]
\setbeamertemplate{itemize subsubitem}[circle]
\setbeamercolor{itemize item}{fg = black}
\setbeamercolor{itemize subitem}{fg = black}
\setbeamercolor{itemize subsubitem}{fg = black}
\setbeamercolor{enumerate item}{fg = black}
\setbeamercolor{enumerate subitem}{fg = black}
\setbeamercolor{enumerate subsubitem}{fg = black}
\setbeamerfont{itemize/enumerate body}{}
\setbeamerfont{itemize/enumerate subbody}{size = \normalsize}
\setbeamerfont{itemize/enumerate subsubbody}{size = \normalsize}

% color links
% ------------------------------------------------------------------------------
\usepackage{hyperref}
\definecolor{urls}{RGB}{204,0,0}
\hypersetup{colorlinks=true, citecolor = urls, urlcolor = urls}

% additional math commands
% ------------------------------------------------------------------------------
\usepackage{bm}                                         % bold math symbols
\newcommand{\niton}{\not\owns}


% additional mathematical operators
% ------------------------------------------------------------------------------
\DeclareMathOperator*{\argmax}{arg\,max}
\DeclareMathOperator*{\argmin}{arg\,min}
\newcommand{\ups}{\upsilon}

% text highlighting
% ------------------------------------------------------------------------------
\usepackage{soul}
\makeatletter
\let\HL\hl
\renewcommand\hl{%
  \let\set@color\beamerorig@set@color
  \let\reset@color\beamerorig@reset@color
  \HL}
\makeatother

% text box colors
\renewcommand\fbox{\fcolorbox{middlegrey}{white}}
\newcommand{\fboxfill}{\fcolorbox{grey}{grey}}

% equation highlighting
% -----------------------------------------------------------------------------
\newcommand{\highlight}[2][yellow]{\mathchoice%
  {\colorbox{#1}{$\displaystyle#2$}}%
  {\colorbox{#1}{$\textstyle#2$}}%
  {\colorbox{#1}{$\scriptstyle#2$}}%
  {\colorbox{#1}{$\scriptscriptstyle#2$}}}%
  
